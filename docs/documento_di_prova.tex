\documentclass{report} % Classe del documento che creiamo. "article" per documenti normali, "beamer" per slides, "report" per usare i capitoli
\usepackage[utf8]{inputenc}
\usepackage[italian]{babel}

\title{Il mio primo documento} % Titolo del documenti
\author{Giacomo Giora} % Autore
\date{15 Ottobre 2025} 

\begin{document} % Inizio corpo del testo

\maketitle % Stampo il titolo nel documento
\tableofcontents % Stampo l'indice dei capitoli

\chapter{Primo capitolo} % Creo un capitolo
\section{Sezione 1} % Con le sue sottosezioni
\section{Sezione 2}
\subsection{Sotto sezione 1} % Sotto sezioni ulteriori
\subsubsection{Sotto sotto sezione} % Sotto sottosezione ulteriore (non compaiono nell'indice e non hanno numerazione)
\paragraph{Sotto sezione paragrafo} % Simile alle sotto sotto sezioni
\subsection{Sotto sezione 2}

\section{Sezione 3}
\begin{itemize} % Elenco puntato "itemize", "enumerate" elenco numerato
    \item Punto 1 % elemento di un elenco
    \item mela
    \item pera
\end{itemize}

\chapter{Secondo capitolo}

% \section{Introduzione}

\end{document}
